\documentclass[12pt, a4paper, oneside]{amsproc}

\usepackage[utf8]{inputenc}
%\usepackage[margin=1.5cm]{geometry}
\usepackage{etoolbox}
\usepackage{dsfont}
\usepackage{xcolor}
\usepackage{cmbright}
\usepackage[cal=rsfso]{mathalfa}
\usepackage{tikz-cd}

\renewcommand{\baselinestretch}{1.1}
\setlength{\parindent}{0pt}
\setlength{\parskip}{5pt}

\pagestyle{plain}

%%%%%%%%%%%%%%%%%%%%%%%%%%%%%%%%%%%%%%%%%%%%%%%%%%%%%%%%%%%

\title{Hopf Algebras}
\author{Alberto Ruiz de Alarcón}
\date{\today}

%%%%%%%%%%%%%%%%%%%%%%%%%%%%%%%%%%%%%%%%%%%%%%%%%%%%%%%%%%%

\begin{document}

\maketitle

\section{What is an algebra?}

Let $(\mathcal A,+,\,\cdot\,)$ be a vector space over a field $\mathds K$,

endowed with a ``multiplication'' $m:\mathcal A\otimes \mathcal A \to \mathcal A$

and a ``unit'' element $1_{\mathcal A}$, or equivalently, $u:\mathds K\to \mathcal A$.

Suppose the multiplication is associative:

\begin{center}
\begin{tikzcd}
\mathcal A\otimes \mathcal A\otimes \mathcal A \arrow[dd, "\mathrm{Id}\otimes m"'] \arrow[rr, "m\otimes\mathrm{Id}"] &  & \mathcal A\otimes \mathcal A \arrow[dd, "m"] \\
                                                                                          &  &                            \\
\mathcal A\otimes \mathcal A \arrow[rr, "m"']                                                               &  & \mathcal A                         
\end{tikzcd}
\end{center}
and it is...
\begin{center}
...
\end{center}

It is  said that $(\mathcal A,+,\,\cdot\,,m,u)$ is an algebra (with unit) (or a $\mathds K$-algebra).
             
\section{The ``dual'' concept: coalgebras}

Let $(\mathcal C,+,\,\cdot\,)$ be a vector space over a field $\mathds K$

endowed with a ``comultiplication'' $\Delta:\mathcal C\to  \mathcal C\otimes \mathcal C$

and a ``counit'' $\epsilon:\mathcal C\to\mathds K$

It is said then that $(\mathcal C,+,\,\cdot\,,\Delta,\epsilon)$ is a co-algebra.

\section{Bilinear algebras}

\section{Hopf algebras}

Let $(\mathcal B,+,\,\cdot\,,m,u,\Delta,\epsilon)$ be a bilinear algebra.

Suppose there is $S:\mathcal B\to\mathcal B$ such that the following diagram commutes:

\begin{center}
\begin{tikzcd}
                                                                             & \mathcal B\otimes \mathcal B \arrow[rr, "\mathrm{Id}_{\mathcal B}\otimes S"] &                           & \mathcal B\otimes\mathcal B \arrow[rd, "m"]   &            \\
\mathcal B \arrow[rd, "\Delta"'] \arrow[ru, "\Delta"] \arrow[rr, "\epsilon"] &                                                                              & \mathds K \arrow[rr, "u"] &                                               & \mathcal B \\
                                                                             & \mathcal B\otimes\mathcal B \arrow[rr, "S\otimes\mathrm{Id}_{\mathcal B}"]   &                           & \mathcal B\otimes \mathcal B \arrow[ru, "m"'] &           
\end{tikzcd}
\end{center}

The map $S$ is then called an antipode.

It is said then that $(\mathcal B,+,\,\cdot\,,m,u,\Delta,\epsilon,S)$ is a Hopf algebra.


In general, $S$ is an antihomomorphism, so $S^2$ is a homomorphism, which is therefore an automorphism if $S$ was invertible (as may be required). 

\subsection*{Involutive Hopf algebras}
If $S^2 = \mathrm{Id}_{\mathcal B}$, then it is said to be an involutive Hopf algebra.

\subsection*{The antipode is unique if it exists}

If a bialgebra admits an antipode, then it is unique. Thus, the antipode does not pose
any extra structure which we can choose: \emph{being a Hopf algebra is a property of a bialgebra}.

\end{document}
